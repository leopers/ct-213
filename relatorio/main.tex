\documentclass[a4paper,12pt]{article}

% Configuração de idioma e codificação
\usepackage[utf8]{inputenc}
\usepackage[T1]{fontenc}
\usepackage[brazil]{babel}

% Layout da página
\usepackage[a4paper, margin=0.8in]{geometry}
\usepackage{setspace}
\setstretch{1.5} % Espaçamento entre linhas

% Fontes e formatação
\usepackage{kpfonts}
\usepackage{amsmath, bm} % Pacotes matemáticos
\usepackage{graphicx} % Inclusão de imagens
\usepackage{caption} % Estilização de legendas
\usepackage{fancyhdr} % Personalização de cabeçalhos e rodapés
\usepackage{titlesec} % Personalização de títulos de seção
\usepackage{xcolor} % Cores para textos e seções
\usepackage{hyperref} % Links clicáveis
\usepackage{background} % Fundo para a página de título
\usepackage{placeins} % Controle de posicionamento de floats (FloatBarrier)
% Ensure the package is loaded correctly for \floatbarrier

\hypersetup{
    colorlinks=true,
    linkcolor=red,
    urlcolor=red,
    citecolor=red
}

% Personalização dos títulos
\titleformat{\section}{\Large\bfseries\color{black}}{\thesection}{1em}{}
\titleformat{\subsection}{\large\bfseries\color{black}}{\thesubsection}{1em}{}

% Cabeçalhos e Rodapés
\pagestyle{fancy}
\fancyhf{}
\fancyhead[R]{\includegraphics[width=2cm]{ITA.png}} % Logo no topo direito
\fancyhead[L]{\textbf{Instituto Tecnológico de Aeronáutica (ITA)}}
\fancyfoot[L]{Leonardo Peres Dias}
\fancyfoot[R]{\thepage}

% Informações do título
\title{
    \textbf{Inteligência Artificial para Robótica Móvel CT-213}\\
    \Large Instituto Tecnológico de Aeronáutica 

    \textbf{Relatório do Laboratório 7 - \textit{Imitation Learning} com Keras}\\
}
\author{
    Leonardo Peres Dias 
}
\date{\today}

% Configuração do fundo (marca d'água apenas na primeira página)
\backgroundsetup{
    scale=1.5,
    color=black,
    opacity=0.2,
    angle=0,
    position=current page.south,
    vshift=5cm,
    hshift=0cm,
    contents={\includegraphics[width=8cm]{ITA.png}}
}

% Início do Documento
\begin{document}

% Aplicar o fundo apenas na primeira página
\BgThispage
\maketitle
\thispagestyle{empty} % Sem cabeçalho/rodapé na página de título

%\begin{abstract}
%Este documento apresenta o relatório do Projeto CES-30 - 2024, desenvolvido com base na segunda forma descrita no enunciado do exame. O projeto abrange tarefas de \textbf{mineração de dados} e \textbf{construção de grafos de conhecimento}, com a aplicação de técnicas específicas para análise e solução prática de problemas reais.
%\end{abstract}

\newpage
\NoBgThispage % Desativa a marca d'água para as páginas seguintes

\tableofcontents

\newpage
\NoBgThispage % Desativa a marca d'água para as páginas seguintes

\section{Breve Explicação em Alto Nível da Implementação}

A rede neural foi implementada utilizando a API Keras. Nesta implementação, foi utilizado um modelo feedforward composto por três camadas densas. A primeira camada possui 75 neurônios com ativação linear e recebe um vetor de entrada unidimensional. Em seguida, aplica-se a função de ativação Leaky ReLU com um parâmetro de inclinação de 0.01.

Posteriormente, a segunda camada densa, composta por 50 neurônios, também utiliza uma ativação linear seguida de uma camada Leaky ReLU. Por fim, a terceira camada densa, com 20 neurônios sem a aplicação de uma função de ativação adicional (ativação linear).

O modelo utiliza o otimizador Adam com o objetivo de minimizar a função de custo definida pelo erro quadrático médio (MSE). Durante o treinamento, o conjunto de dados é fornecido ao modelo em um único batch correspondentes ao tamanho total do dataset.

\newpage

\section{Figuras Comprovando Funcionamento do Código}


\subsection{Função de Classificação \texttt{sum\_gt\_zeros}}

\subsubsection{Sem regularização}

\begin{figure}[!h]
    \centering
    % Primeira linha: duas figuras lado a lado
    \begin{minipage}[b]{0.45\linewidth}
        \centering
        \includegraphics[width=\linewidth]{figures/convergence_sgz_l0.0.eps}
        \caption{Convergência do custo}
    \end{minipage}
    \hfill
    \begin{minipage}[b]{0.45\linewidth}
        \centering
        \includegraphics[width=\linewidth]{figures/dataset_sgz_l0.0.eps}
        \caption{Dataset}
    \end{minipage}
    
    \vspace{0.5cm} % Espaçamento entre as linhas
    
    % Segunda linha: figura centralizada
    \includegraphics[width=0.9\linewidth]{figures/nn_classification_sgz_l0.0.eps}
    \caption{Tarefa de classificação}
    
    \label{fig:three_figures}
\end{figure}

\subsubsection{Com regularização}

\begin{figure}[!h]
    \centering
    % Primeira linha: duas figuras lado a lado
    \begin{minipage}[b]{0.45\linewidth}
        \centering
        \includegraphics[width=\linewidth]{figures/convergence_sgz_l0.002.eps}
        \caption{Convergência do custo}
    \end{minipage}
    \hfill
    \begin{minipage}[b]{0.45\linewidth}
        \centering
        \includegraphics[width=\linewidth]{figures/dataset_sgz_l0.002.eps}
        \caption{Dataset}
    \end{minipage}
    
    \vspace{0.5cm} % Espaçamento entre as linhas
    
    % Segunda linha: figura centralizada
    \includegraphics[width=0.9\linewidth]{figures/nn_classification_sgz_l0.002.eps}
    \caption{Tarefa de classificação}
    
    \label{fig:three_figures}
\end{figure}

\newpage
\subsection{Função de Classificação XOR}

\subsubsection{Sem regularização}

\begin{figure}[!h]
    \centering
    % Primeira linha: duas figuras lado a lado
    \begin{minipage}[b]{0.45\linewidth}
        \centering
        \includegraphics[width=\linewidth]{figures/convergence_xor_l0.0.eps}
        \caption{Convergência do custo}
    \end{minipage}
    \hfill
    \begin{minipage}[b]{0.45\linewidth}
        \centering
        \includegraphics[width=\linewidth]{figures/dataset_xor_l0.0.eps}
        \caption{Dataset}
    \end{minipage}
    
    \vspace{0.5cm} % Espaçamento entre as linhas
    
    % Segunda linha: figura centralizada
    \includegraphics[width=0.9\linewidth]{figures/nn_classification_xor_l0.0.eps}
    \caption{Tarefa de classificação}
    
    \label{fig:three_figures}
\end{figure}

\newpage

\subsubsection{Com regularização}

\begin{figure}[!h]
    \centering
    % Primeira linha: duas figuras lado a lado
    \begin{minipage}[b]{0.45\linewidth}
        \centering
        \includegraphics[width=\linewidth]{figures/convergence_xor_l0.002.eps}
        \caption{Convergência do custo}
    \end{minipage}
    \hfill
    \begin{minipage}[b]{0.45\linewidth}
        \centering
        \includegraphics[width=\linewidth]{figures/dataset_xor_l0.002.eps}
        \caption{Dataset}
    \end{minipage}
    
    \vspace{0.5cm} % Espaçamento entre as linhas
    
    % Segunda linha: figura centralizada
    \includegraphics[width=0.9\linewidth]{figures/nn_classification_xor_l0.002.eps}
    \caption{Tarefa de classificação}
    
    \label{fig:three_figures}
\end{figure}

\newpage

\subsection{\textit{Imitation Learning}}

\begin{figure}[htbp]
    \centering
    % Primeira linha: duas imagens lado a lado
    \begin{minipage}[b]{0.45\linewidth}
        \centering
        \includegraphics[width=\linewidth]{figures/rightAnklePitch.eps}
        \label{fig:image1}
    \end{minipage}
    \hfill
    \begin{minipage}[b]{0.45\linewidth}
        \centering
        \includegraphics[width=\linewidth]{figures/rightAnkleRoll.eps}
        \label{fig:image2}
    \end{minipage}
    
    \vspace{1em} % Espaço entre as linhas
    
    % Segunda linha: três imagens lado a lado
    \begin{minipage}[b]{0.3\linewidth}
        \centering
        \includegraphics[width=\linewidth]{figures/rightHipPitch.eps}
        \label{fig:image3}
    \end{minipage}
    \hfill
    \begin{minipage}[b]{0.3\linewidth}
        \centering
        \includegraphics[width=\linewidth]{figures/rightHipRoll.eps}
        \label{fig:image4}
    \end{minipage}
    \hfill
    \begin{minipage}[b]{0.3\linewidth}
        \centering
        \includegraphics[width=\linewidth]{figures/rightKneePitch.eps}
        \label{fig:image5}
    \end{minipage}
    
    \caption{Movimento das juntas do robô obtidas pela rede neural.}
    \label{fig:cinco_imagens}
\end{figure}

\section{Discussões}

\subsection{Regularização}

Nota-se que o emprego de regularização L2 na tarefa de classificação \texttt{sum\_gt\_zeros} não trouxe resultados relevantes, uma vez que o dataset é linearmente separável. No entanto, na tarefa de classificação XOR, a regularização L2 demonstrou um desempenho superior em comparação à ausência de regularização, visto que previniu a rede de realizar \textit{overfitting} e, assim, perder poder de generalização.


\subsection{\textit{Imitation Learning}}

A rede neural foi capaz de aprender o movimento, reproduzindo o movimento das juntas do robô. A comparação entre os movimentos reais e os previstos pela rede neural mostra que a rede conseguiu capturar o padrão de movimento desejado, embora com algumas variações.
\end{document}